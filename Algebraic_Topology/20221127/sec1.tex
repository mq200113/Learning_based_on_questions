\section{Definitions}
Prove that $f^{-1}(A)=f_{|C}^{-1}(A)\cup f_{|K}^{-1}(A)$.

\noindent Suppose $x\in f^{-1}(A)$ then $f(x)\in A$. If $x\in C$, then $f(x)=f_{|C}(x)\in A$ so $x\in f^{-1}_{|C}(A)$. Likewise for $x\in K$. Suppose $x\in f_{|C}^{-1}(A)\cup f_{|K}^{-1}(A)$. If $x\in f_{|C}^{-1}(A)$, $f_{|C}(x)=f(x)\in A$. Likewise for $x\in f_{|K}^{-1}(A)$.

In a compact metric space, we can associate a Lebesgue number for every open covering. We can assume for every $\delta$, there exists $x\in X$, for any $\alpha\in A$, $B_{\delta}(x)\not\subset U_{\alpha}$. So we choose $\delta=1/n$ and $x$ as $x_n$ such that $B_{1/n}(x_n)$ is not contained in any $U_{\alpha}$.

The term “space obtained by attaching an $n$-cell to $X$ along $f$”  sees no motiviation. Importantly, I don't see what is an $n$-cell. Since the notation is $X\cup_f D^n$, can I trivially consider $D^n$ i.e. the $n$ dimensional unit disk, as the so-called $n$-cell?

The understanding of this definition means an equivalence class $(X\sqcup D^n)\backslash \sim$. First why $X$ is disjoint with $D^n$? Next how should I understand the equivalence relation generated by $x\sim f(x)$ for all $x\in S^{n-1}\subset D^n$? In this argument, I see two problems: first please prove that $S^{n-1}\subset D^n$, second if $x\sim f(x)$ what about $f(x)$ and $f(f(x))$? Note that $f:S^{n-1}\longrightarrow X$, so $f(x)\in X$ and $f(x)\not \in D^{n}$ as is $S^{n-1}$. Thus $f(x)\not \sim f(f(x))$. Show me some typical elements in $(X\sqcup D^n)\backslash \sim$? For example, if $x\in X\backslash f(S^{n-1})\sqcup x\in D^n\backslash S^{n-1}$, then $\left\{x\right\}$ is an element in it; if $x\in S^{n-1}$, then $\left\{x,f(x)\right\}$ is an element in it. Suppose without the part $S^{n-1}$, we are actually not touch with a part of $X$,  say $X\backslash f(S^{n-1})$ and the disk without boundry, say $D^n\backslash S^{n-1}$. But we when it comes to the boundry, we stick the sphere, under the guide of $f$, to the subset of $X$, say $f(S^{n-1})$.

How can I understand the process of obtaining $X^{(n)}$ from $X^{(n-1)}$:
$$
X^{(n)}=\left(X^{(n-1)}\sqcup \bigsqcup_{\alpha}D_{\alpha}^{n}\right)\slash \left\{x\sim f_{\alpha}(x)\right\}
$$
The first key question is that what are $D_{\alpha}^n$? Why do they not intersect with each other? Well I think that these $n$-cells just centered at different center such that they can keep away from each other.

A 1-dimensional ball is an interval, while a 1-dimensional sphere is the sets of two end points.

Hawaiian Earring is an example of a non-cell complex.

The real question is, for example, how can we ensure that when we attach a 1-cell to a single point, we get a circle? Since there is no mechanism during the gluing that gurantees the “form” of the final space. A place that seems to be safe is to implicitly assume that we can achieve this control, at worse, up to some continous transformations.
