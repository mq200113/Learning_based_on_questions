\section{Smooth Functions on a Euclidean Space}

How many partial derivatives we need to examine according to the definition of $C^{k}$ for a real-valued function? Since $$
\frac{\partial^j f}{\partial x^{i_1}\cdots\partial x^{i_j}}
$$
for all $j\le k$ must be examined, for each $j$ we have $n^j$ choices, and the total choice is:
$$
\sum_{j=0}^{k}n^{j}=\frac{1-n^{k+1}}{1-n}
$$
So every increase in $k$ will increase exponentially the work.

What is the idea behind the example 1.3, where we try to find a $C^{\infty}$ function which is not real-analytic? We want this function's derivatives of all order to be zero, at a point, say ‘flat', but does that means no function but a constant function can achieve that at this flat point? The example gives a counterexample:
$$
f(x)=\left\{\begin{aligned}
e^{-1/x}&\text{ for }x>0\\
0&\text{ for }x\le 0\\
\end{aligned}\right.
$$ 

We first show by induction that for $x>0$ and $k\ge 0$, the $k$th derivative $f^{(k)}(x)$ is of the form $p_{2k}(1/x)e^{-1/x}$ for some polynomial $p_{2k}(y)$ of degree $2k$ in $y$. Suppose that the property is valid for $k$, we show that it is also valid for $k+1$.
$$
f^{(k+1)}=e^{-\frac{1}{x}}\left[\frac{1}{x^2}\left(p_{2k}(\frac{1}{x})-p_{2k}^{'}(\frac{1}{x})\right)\right]
$$
We can identify $\frac{1}{x^2}\left(p_{2k}(\frac{1}{x})-p_{2k}^{'}(\frac{1}{x})\right)$ as a polynomial $p_{2(k+1)}(\frac{1}{x})$. Then since $$
\lim\limits_{x\to 0^{+}}p_{2k}(1/x)e^{-1/x}=0
$$
for all $k$, we have $f^{k}(0)=0$.