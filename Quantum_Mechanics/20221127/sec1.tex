\section{Two level problems}
What's special of a two levels system? I see two eigenstates and two eigenvalues of the Hamiltonian, but in a system, say the harmonic oscillator, we can have infinite eigenstates and infinite eigenvalues, so why say the harmonic oscillator is a one-level system while it has more eigenstates and eigenvalues than the two-level system? I think that two-channel means only have two eigenstates, while for one-level, I tolerate it to have infinite eigenstates.

Why “perturbative” term should not depend on time?

Why shall we first make a unitary transformation to change basis from $\bket{\psi_{i}^{0}}$ to $\bket{\psi_i}$?

Will the assumption that $H_{12}^{'}$ is real make no loss of generality? 

We get the energy from the solvability of a system of equations, that is from the fact that its determinant equals to zero, i.e., if a system is solvable its matrix cannot be invertible, since for an invertible matrix, its determinant times its inverse's determinant equal to 1, this cannot be achieved if its determinant is equal to 0.

In which case can we have $H_{11}^{'}=0=H_{22}^{'}$ and $E_1^{0}=0=E_2^{0}$?

Why should we choose the “+” sign so that for $\Delta>0$ and $D>0$ we will have $\theta=0$ in the limit when $D\longrightarrow 0$? 
$$
\tan\theta\simeq \frac{-\Delta+\Delta+\Delta2\left(\frac{D}{\Delta}\right)^2}{2D}=\frac{D}{\Delta}
$$

What is the connection between the $(E_{1},E_{2})$ and $(\bket{\psi_{1}},\bket{\psi_{2}})$ calculated in the formalism in terms of mixing angles and the $(E_{+},E_{-})$ and $(\bket{\psi_{+}},\bket{\psi_{-}})$ in the basic formalism? I know that the calculation is tedious. What's the interest of the formalism in terms of mixing angles?