\section{Background noise}
In order to understand why we say that the noise obeys the Gaussian distribution, you have to turn your head to the left side and focus on the amplitude, rather than the frequency.

I don't understand why we use $\gamma(f)$ to quantifiy the noise? What is its physical meaning? And I notice that a diode is said to be equivalent to two intersect circles where we label $i_g^2$ while for the resistance we put a circle in series with it and labels $e_R^2$. These labels are said to represent the power of the noise. And we also use the $\gamma$ in a later section to represent the spectral density of power.

There is also a model for the operational amplifier, which I have no idea why it looks like that? And I don't know how to analyze the noise, for example, what physical quantities should I look at first? Besides all this, it seems that I need to develop a skill in how to look at the instruction book of the operational amplifier, especially how to pick up the information to calculte some quantities relevant to noise from a graph and a table?

