%\usepackage{ctex}
\usepackage{amsfonts}
\usepackage{amsmath}
\DeclareMathOperator\arcsinh{arcsinh}
\usepackage{amssymb}


\usepackage{amsthm}%%这个宏包定义了如何构建不同风格的定理环境
%%一般的模式:先用\newtheoremstyle去定义一个风格,然后调用这个\theoremstyle,下方定义一些\newtheorem属于这个风格。
%%官方的宏包写的很清楚
\newtheoremstyle{mystyle}
{2ex}% above space
{2ex}% below space
{}% body font
{}% indent amount
{\bfseries}% head font
{.}% post head punctuation
{ }% post head punctuation
{}% head spec

\theoremstyle{mystyle}
\newtheorem{tuilun}{Corollary}
\newtheorem{dingyi}{Definition}
\newtheorem{lizi}{Example}
\newtheorem{lianxi}{Exercise}
\newtheorem{yinli}{Lemma}
\newtheorem{jihao}{Notation}
\newtheorem{yueding}{Convention}
\newtheorem{mingti}{Proposition}
\newtheorem{dingli}{Theorem}
\newtheorem{caogao}{Scratch}
\newtheorem{pinglun}{Remark}



\renewenvironment{proof}{{\bf\noindent Proof: }}{\hfill $\blacksquare$\par}

\newenvironment{jie}{{\bf\noindent Solution: }}{\hfill $\blacksquare$\par}


\newcommand{\zsqj}[1]{\llbracket #1 \rrbracket}%整数区间
\newcommand{\piandao}[2]{\frac{\partial #1}{\partial #2}}%偏导
\newcommand{\fengshu}[2]{\frac{#1}{#2}}%分数
\newcommand{\ybds}[2]{\frac{\mathrm{d} #1}{\mathrm{d} #2}}%一般导数
\newcommand{\dwf}{\mathrm{d}}%d微分
%常见集合符号
\newcommand{\Cc}{\mathbb{C}}
\newcommand{\N}{\mathbb{N}}
\newcommand{\K}{\mathbb{K}}
\newcommand{\Nn}{\mathbb{N^{*}}}
\newcommand{\Q}{\mathbb{Q}}
\newcommand{\R}{\mathbb{R}}
\newcommand{\Z}{\mathbb{Z}}

\newcommand{\rw}[1]{\shadowsize=8pt
	\shadowbox{\parbox{\textwidth}{Review: #1}}}%加框
\newcommand{\jk}[1]{\shadowsize=8pt
	\shadowbox{\parbox{\textwidth}{#1}}}%加框

%括号
\newcommand{\bket}[1]{\left\lvert #1\right\rangle}
\newcommand{\brak}[1]{\left\langle #1 \right\rvert}
\newcommand{\braket}[2]{\left\langle #1\middle\vert #2 \right\rangle}
\newcommand{\bra}{\langle}
\newcommand{\ket}{\rangle}

\usepackage{booktabs}
\usepackage{caption}
\usepackage{soul}
\usepackage{color, xcolor}
\usepackage{enumitem}
\usepackage{esint}
\usepackage{fancyhdr}
\usepackage{float}
\usepackage{framed}
\usepackage{graphicx}
\usepackage{listings}
\usepackage{mathdots}
\usepackage{mathtools}
\usepackage{mathrsfs}
\usepackage{marginnote}
\usepackage{microtype}
\usepackage[version=4]{mhchem}
\usepackage{multirow}
\usepackage{pdflscape}
\usepackage{pgfplots}
\usepackage{siunitx}
\usepackage{slashed}
\usepackage{stmaryrd}
\usepackage{subfig}
\usepackage{tabularx}
\usepackage{tcolorbox}
\tcbuselibrary{skins, breakable, theorems}
\usepackage{textcomp}
\usepackage{tikz}
\usepackage{tkz-euclide}
\usepackage{cancel}
\usepackage{wrapfig}
\usepackage[normalem]{ulem}
\usepackage[all]{xy}
\usepackage{cmbright}%我觉得顺眼的字体
\usepackage{fancybox}%做各种形式的盒子
\usepackage{imakeidx}%做索引
\makeindex
\usepackage[
a5paper,
rmargin=10pt,lmargin=10pt,
tmargin=0pt,bmargin=10pt,
]{geometry}

\usepackage[hidelinks,
pdfauthor={Philippe MENG},
pdfsubject={Notes: \ncourse\ -\ \myedition},
pdftitle={\ncourse\ -\ \myedition},
pdfkeywords={\ndates\ \nyear\ \ncourse}]{hyperref}
\title{\ncourse \\ {\small Edition: \myedition}}
\pgfplotsset{compat=1.12}
\pagestyle{plain}%原本为fancyplain,会有页眉
%\lhead{\emph{\nouppercase{\leftmark}}}
%\rhead{\ifnum\thepage=1\else\ncourse\fi}

%标题样式设置
\usepackage{titlesec}
\titleformat{\chapter}[block]{\LARGE\bfseries}{Chapter \arabic{chapter}}{1em}{}[]
\titlespacing{\chapter}{0pt}{-2pt}{2pt}


%插入pdf_tex
\usepackage{transparent}
\usepackage{import}
\newcommand{\dizhi}{C:/Users/10109/Desktop/Notes}%注意Windows下地址正常用\隔开,但在tex中要换成/
\newcommand{\includesvg}[3]{%
	\def\svgwidth{#1}%
	\import{\dizhi/#2/}{#3.pdf_tex}%
}

\newcounter{handout}[section]
\newcommand{\ho}[1]{\addtocounter{handout}{1}\href{\dizhi/#1}{Handout \thehandout.}}

\newcounter{trash}[section]
\newcommand{\rtd}{\addtocounter{trash}{1}{{\color{red}{Rest to be done}} \thetrash !!!}}

\newcommand{\kh}{\vspace{20pt}}
\newcommand{\yd}{$\circ$}