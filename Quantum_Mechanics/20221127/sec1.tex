\section{Two level problems}
What's special of a two levels system? I see two eigenstates and two eigenvalues of the Hamiltonian, but in a system, say the harmonic oscillator, we can have infinite eigenstates and infinite eigenvalues, so why say the harmonic oscillator is a one-level system while it has more eigenstates and eigenvalues than the two-level system? I think that two-channel means only have two eigenstates, while for one-level, I tolerate it to have infinite eigenstates.

Why “perturbative” term should not depend on time?

Why shall we first make a unitary transformation to change basis from $\bket{\psi_{i}^{0}}$ to $\bket{\psi_i}$?

Will the assumption that $H_{12}^{'}$ is real make no loss of generality? 

We get the energy from the solvability of a system of equations, that is from the fact that its determinant equals to zero, i.e., if a system is solvable its matrix cannot be invertible, since for an invertible matrix, its determinant times its inverse's determinant equal to 1, this cannot be achieved if its determinant is equal to 0.

In which case can we have $H_{11}^{'}=0=H_{22}^{'}$ and $E_1^{0}=0=E_2^{0}$?

Why should we choose the “+” sign so that for $\Delta>0$ and $D>0$ we will have $\theta=0$ in the limit when $D\longrightarrow 0$? 
$$
\tan\theta\simeq \frac{-\Delta+\Delta+\Delta2\left(\frac{D}{\Delta}\right)^2}{2D}=\frac{D}{\Delta}
$$

What is the connection between the $(E_{1},E_{2})$ and $(\bket{\psi_{1}},\bket{\psi_{2}})$ calculated in the formalism in terms of mixing angles and the $(E_{+},E_{-})$ and $(\bket{\psi_{+}},\bket{\psi_{-}})$ in the basic formalism? I know that the calculation is tedious. What's the interest of the formalism in terms of mixing angles?

Why $A=B$ is called the degenerate case?

Why when $\theta$ is close to but slightly less than $\pi/2$, we shall see that $\bket{\psi_{+}}\simeq \bket{\psi_{2}^{0}}$ and $\bket{\psi_{-}}\simeq -\bket{\psi_{1}^{0}}$?

How to appreciate the name “see-saw” mechanism in neutrino physics for $\epsilon=\sqrt{\frac{m_1}{m_2}}$? What does the condition $A=0,D\ll B$ means?

What do we mean by saying that when $D\not =0$ these levels “repel” each other?

Why is it important to look at the behavior of $E_{\pm}$ and $\bket{\psi_{\pm}}$ as $\Delta \to\pm\infty$?

What we say that the eigenvalues of
$$
\frac{1}{2}\begin{bmatrix}
E_1+E_2-\delta \cos2\theta&-\delta \sin2\theta\\
-\delta \sin2\theta&E_1+E_2+\delta \cos2\theta
\end{bmatrix}
$$
are what defined as $E_{+}$ and $E_{-}$?

Why can we have different time-evolution equations for $\bket{\phi(t)}$ versus $\bket{\phi_1(t)}$ and $\bket{\phi_2(t)}$? A two-level Hilbert space with time dependence involved is not the same space as we considered above when there is no time dependency.

Why we have 
$$
i\hbar \left(\dot{c_1}\bket{\psi_{1}^{0}(t)}+\dot{c_2}\bket{\psi_{2}^{0}(t)}\right)=c_1(t)H^{'}(t)\bket{\psi_{1}^{0}(t)}+c_2(t)H^{'}(t)\bket{\psi_{2}^{0}(t)}
$$
when multiplying by $\brak{\psi_{1}^{0}(t)}$ is:
$$
i\hbar \dot{c_1}=c_1H_{11}^{'}(t)+c_2H^{'}_{12}(t)\exp(i\omega_{12}t),\omega_{12}=\frac{(E_{1}^{0}-E_{2}^{0})}{\hbar}
$$
Because I ignored an important information:
$$
H_{ij}^{'}(t)=\brak{\psi_{i}^{0}(0)}{H^{'}(t)}\bket{\psi_{j}^{0}(0)}
$$

In which place the explicit time dependence in $H^{'}$ makes the results from the previous sections involving the mixing angles fail?

Isn't it strange to see a time-dependence resonance with zero amplitude at some time? Why do the resonance important?

Why we simply say that $\boldsymbol{d}$ in the $x$-direction and then conclude $H^{'}=-\boldsymbol{d}.\boldsymbol{E}\sim x$?

What is the symmetry consideration that makes $\brak{\psi_S}H^{'}\bket{\psi_S}=0=\brak{\psi_A}H^{'}\bket{\psi_A}$ and $\brak{\psi_S}H^{'}\bket{\psi_A}\not=0$?

Why should the ammonia molecules in the state $\bket{\psi_A}$ enter the apparatus during the emission cycle? 

I can't understand what is the tunneling of the middle barrier. What is the rate of an emission?

\section{Spin $1/2$ systems in the presence of magnetic fields}
Why spin $1/2$ systems are intrinsically two-level systems? How does the two states in the spin-space be compatible with the infinite possible states in the position space?

Why $H(t)\bket{z+(t)}=\frac{1}{2}\hbar\omega_0\bket{z+(t)}$?

How can we say that when the applied frequency equals
the precession frequency, in a sense the spin now sees a constant magnetic field?

Why do we make the changes in the Hamiltonian
$$
\hbar \bar{\omega_0}\longrightarrow \hbar\bar{\omega_0}\cos\theta,\hbar\bar{\omega_1}\longrightarrow\hbar\bar{\omega_1}\sin\theta
$$
when we consider the case where the spin is oriented in the same direction as the magnetic field?

\section{Oscillation and regeneration in neutrinos and neutral K-mesons as two-level systems}

Why does the matrix representation of the total Hamiltonian for the neutrinos is given by
$$
\begin{bmatrix}
\brak{\nu_1}&\brak{\nu_2}
\end{bmatrix}
\begin{bmatrix}
m_1^2&0\\
0&m_2^2
\end{bmatrix}\begin{bmatrix}
\bket{\nu_1}\\
\bket{\nu_2}
\end{bmatrix}+\brak{\nu_e}H^{'}\bket{\nu_e}
$$

What is the meaning of the adiabatic variation of eigenvalues?

What does it mean by saying that electrons is boosted to the upper curve and the muon is demoted to the lower one?