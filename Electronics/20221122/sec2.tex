\section{Combinoric and sequential logic}
When I see the logic functions for a while, I don't understand why there needs to list so many functions, I can go dirctly to the Boolean algebra without even have the physical representation as follows: a variable is equivalent to a switch, which is originally connected, it is when you press it that it becomes cut-off, but when you put a bar on it, the scheme is reversed, the switch is originally cut-off, while you have to pull it up to make it connected.

Now what are the new rules for Boolean algebra, which have two operations with symbol like $.$ and $+$. This can be a big abuse of symbols if the rules for them are only partially inherited from the multiplication and addition. I list some abnormal properties as follows:

\noindent 1. $a+(b.c)=(a+b).(a+c)$

\noindent 2. $a+1=1$

\noindent 3. $a.\overline{a}=0$

\noindent 4. $a+\overline{a}=1$

\noindent 5. $a.a=a$

\noindent 6. $a+a=a$

\noindent 7. $\overline{a.b}=\overline{a}+\overline{b}$

\noindent 8. $\overline{a+b}=\overline{a}.\overline{b}$

It comes to me that I don't understand how to see the table of Karnaugh. It seems to me that 0 in this table means the variable must take a bar on it and the column is put in front of the line.

And there is another notion: teeterboard. What can make use of it?